%数学函数
\documentclass[UTF8]{ctexart}
\title{The mathetical equation}
\author{LiChao}
\date{June 15.2012}
\usepackage{amsmath}

\begin{document}
\maketitle
\part{some simple equation}
\section{The fraction}
This is a equation with a fraction.The command is similar to the mathematical formula except that the way to input the fraction.
\[
  \frac{1}{x}+\frac{1}{y+z}=\frac{1}{f}
\]

\section{The square root}
This is a equation with a square root.The command is also simliar to the mathematical formula except that the way to input the square root.
\[
  \sqrt{x}+\sqrt{y}=\sqrt{z}
\]

\section{The cosine and so on}
\[
  2\cos \pi x\sin x =\sin 2x.
\]

\section{The limit}
\[
  \lim_{n\to\infty}\frac{1}{n}=0
\]

\section{The integration}
\[
  \int \frac{1}{x} dx=ln|x|+C
\]
The C stands for a constant.

\vspace{2cm}
\part{The partenthesis}
\section{The matching parenthesis}
\[
  \lim_{x\to 0}(1+\frac{1}{x})^x=\mathrm{e}
\]

What is the difference between the two methods to display a parenthesis? By contrast, we know that the size of the parenthesis is different.

\[
  \lim_{x\to 0}\left(1+\frac{1}{x}\right)^x=\mathrm{e}
\]

We can also set the size of the parenthesis by ourselves.Now let us have a try.
\[
  \Bigg< \bigg\{ \Big[ \big( just  have  a  try \big) \Big] \bigg\} \Bigg>
\]

\vspace{3cm}
\part {Multiple line formula}
Sometimes we have several different formulas which in the different lines.

And we must use the package `amsmath'.
\begin{gather}
x+y=5\\
2x+3y+z=6
\end{gather}

We can find that the two formulas don't align.In the next line we will take a measure to solve the prlblem.
And the two formulas will be aligned after the mark `\&'.

\begin{align}
x+y&=5\\
2x+3y+z&=6
\end{align}

There is a mark number after each formula,and I want to get rid of it.

\begin{align*}
x+y&=5\\
5x+2y+3z&=6
\end{align*}

\vspace{2cm}
\part{A long formula}
We maybe come across a phenomenon that we couldn't put a formula in a only line.Faced to this phenomenon,we should use the command `split' which must be put in the the command `equation'.

\begin{equation}
\begin{split}
f(x)=&a_1x^5+b_1x^4+c_1x^3+d_1x^2+x+\\
     &a_2x^5+b_2x^4+c_2x^3+d_2x^2+x+\\
     &2\cdot 1C
\end{split}
\end{equation}

Next,we will have a formula with a big parenthesis.

\begin{equation}
\left.
\begin{gathered}
x+y=5\\
x+y+z=6
\end{gathered}
\right \}
\Rightarrow z=1
\end{equation}

\part{The theorem}
Sometimes,we may want to write a theorem.We may input the theorem in the first line and the corollary in the next line.
\subsection{The first kind}
\newtheorem{theorem}{Theorem}
\newtheorem{corollary}{Corollary}
\begin{theorem}
if a\textgreater b and b\textgreater c,we can get that a\textgreater c.
\end{theorem}
\begin{corollary}
if a\textgreater b and b\textgreater c and c\textgreater d,we can get that a\textgreater d.
\end{corollary}

\begin{theorem}
if a\textless b and b\textless c,we can get that a\textless c.
\end{theorem}
\begin{corollary}
if a\textless b and b\textless c and c\textless d,we can get that a\textless d.
\end{corollary}

\subsection{The second kind}
\newtheorem{thrm}{Theorem}
\newtheorem{corl}[thrm]{Corollary}
\begin{thrm}
if a\textgreater b and b\textgreater c,we can get that a\textgreater c.
\end{thrm}
\begin{corl}
if a\textgreater b and b\textgreater c and c\textgreater d,we can get that a\textgreater d.
\end{corl}

\subsection{The third kind}
\newtheorem{thm}{Theorem}[section]
\newtheorem{cor}[thm]{Corollary}
\begin{thm}
if a\textgreater b and b\textgreater c,we can get that a\textgreater c.
\end{thm}
\begin{cor}
if a\textgreater b and b\textgreater c and c\textgreater d,we can get that a\textgreater d.
\end{cor}
\end{document}