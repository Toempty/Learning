%测试表格的生成
\documentclass[UTF8]{ctexart}
\title{表格}
\author{LiChao}
\date{June 14.2012}

\begin{document}
\maketitle

%表格
我的表格!

\begin{tabular}{|l|c|r|} %l表示左对齐 ,c表示中对齐 , r表示右对齐  |表明对应的位置有竖线
\hline                   %表示对应的位置有横线
左列 & 中列 & 右列 \\     %& 元素之间的分隔     \\行与行之间的分隔
\hline
第二行 & 第二行 & 第二行 \\
\hline
第三行 & 第三行 & 第三行 \\
\hline

\end{tabular}

表格结束!

\begin{tabular}{|l|c|r|}
\hline
左列& 中列& 右列\\
\hline
第二行& 第二行& 第二行\\
\hline
\multicolumn{2}{|c|}{跨越2012} & 第三行\\  %{2}指明了跨越的行数   {|c|} 指明对齐方式和边框线
\hline
第四行& 第四行& 第四行\\
\hline
\end{tabular}


浮动表格!
\begin{table}[htbp!] %,h(here,当前位置)、t(top,页面顶部)、b(bottom,页面底部)、p(page,单独一页)表明允
                     %许将表格放置在哪些位置,而! 表示不管某些浮动的限制。
\centering
\begin{tabular}{|l|c|c|r|}
\hline
第一行& 第一行& 第一行& 第一行\\
\hline
第二行& 第二行& 第二行& 第二行\\
\hline
第三行& 第三行& 第三行& 第三行\\
\hline
第四行& 第四行& 第四行& 第四行\\
\hline
\end{tabular}
\end{table}

结束浮动表格!

\end{document}


