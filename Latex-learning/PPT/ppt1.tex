\documentclass{beamer}
%\usetheme{Antibes}    %带顶栏的
%\usetheme{Darmstadt}
%\usetheme{Frankfurt}
%\usetheme{JuanLesPins}
%\usetheme{Montpellier}
%\usetheme{Singapore}

%\usetheme{Boadilla}  %带底栏的
%\usetheme{Madrid}

%\usetheme{AnnArbor}    %带顶栏和底栏的
%\usetheme{Berlin}
%\usetheme{CambridgeUS}
\usetheme{Copenhagen}    %不错
%\usetheme{Dresden}         %还行
%\usetheme{Ilmenau}         %还行
%\usetheme{Luebeck}         %还行
%\usetheme{Malmoe}          %还行
%\usetheme{Szeged}
%\usetheme{Warsaw}          %还行

%\usetheme{Berkeley}    %带侧栏的       还行
%\usetheme{Goettingen}
%\usetheme{Hannover}
%\usetheme{Marburg}
%\usetheme{PaloAlto}

%外部主题,用以设置是否有顶栏,底栏和侧栏以及它们的结构
%\useouttheme{default}  %default infolines miniframes sidebar smoothbars split shadow tree smoothree

%内部主题,设置正文的内容(标题,列表,定理)的样式
\useinnertheme{rounded}  %default circles rectangles rounded

%颜色主题,设置文稿的各种元素配色
\usecolortheme{default}  %default albatross beaver beetle crane dolphin dove fly lily orchid rose seagull seahorse sidebartab structure whale wolverine。

%字体主题设定演示文稿的字体 
\usefonttheme{default}  %default serif structurebold structureitalicserif  structuresmallcapsserif。
\begin{document}

\title{My first latex PPT}
\author{LiChao}
\date{June 16.1012}

\begin{frame}
  \titlepage
\end{frame}

\begin{frame}
  \frametitle{The menu}
  \tableofcontents
\end{frame}

\section{The factors}
\begin{frame}
\frametitle{The factors}
  \begin{enumerate}
     \item The most important factor \pause
     \item The second most important factor \pause
     \item The last foctor
  \end{enumerate}
\end{frame}

\section{The difference between different commands}
\begin{frame}
\frametitle{The difference between diffrerent commands}
  \begin{enumerate}
     \item <1-> The first step
     \item <2-> The second step
     \item <5-> the fifth step
     \item <3-> the third step
     \item <4-> the fourth step
  \end{enumerate}
\end{frame}
  

\section{Just do it}

\begin{frame}
\frametitle{Just do it}
   \begin{itemize}
      \item What to do?           \pause
      \item Why to do?            \pause
      \item How to do?
   \end{itemize}
\end{frame}
\end{document}